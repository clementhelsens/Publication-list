\documentclass[12pt]{article}
\usepackage{fancyhdr}

%To use pdflatex, uncomment these lines, as well as the \href lines
%in each entry
%\usepackage[pdftex,
%       colorlinks=true,
%       urlcolor=blue,       % \href{...}{...} external (URL)
%       filecolor=green,     % \href{...} local file
%       linkcolor=red,       % \ref{...} and \pageref{...}
%       pdftitle={Papers by AUTHOR},
%       pdfauthor={Your Name},
%       pdfsubject={Just a test},
%       pdfkeywords={test testing testable},
%       pagebackref,
%       pdfpagemode=None,
%        bookmarksopen=true]{hyperref}
%usepackage{arial}
%\renewcommand{\familydefault}{\sfdefault} % San serif
\renewcommand{\labelenumii}{\arabic{enumi}.\arabic{enumii}}
\renewcommand{\labelenumi}{[\arabic{enumi}]}

\pagestyle{empty}
\oddsidemargin 0.0in
\textwidth 6.5in
\topmargin -0.75in
\textheight 9.5in

\begin{document}
\pagestyle{fancy}
\fancyhf{}
\fancyfoot[RO]{\small \tt Cl\'ement Helsens  \space  \space \space \space \space \space \normalsize \textrm{\thepage}}

%%\fancyfoot[RO]{\small \tt Cl\'ement Helsens - Concours CNRS 2013 (01/07)}
%%\thepage

%\title{}
%\author{}
%\date{}
%\maketitle





\begin{center}
%{\bf \Large  Experimental Physicist at CERN}\\
%\vspace{0.8cm}
{\bf \Large List of publications} \\
\end{center}

\vskip 1.0 cm

\noindent Please find below a reduced list of the ten most significant papers and conference notes of particular scientific interest on which I was a primary author with substantial
contributions to the research. Full list of publications available upon request.\\


\noindent [1]  As a Ph.D. student I studied the discovery potential of heavy neutral gauge bosons decaying into dimuon pairs, $Z^{\prime} \rightarrow \mu^{+}\mu^{-}$. 
For this purpose I developed a sophisticated statistical method which was adopted by the exotics physics group. For muons of high momentum, 
the Muon Spectrometer momentum resolution is dominated by the misalignment. I have participated in detailed studies to assess the impact of this effect on $Z^{\prime}$ 
search sensitivity.\\

\noindent [2] With the very first LHC data I actively participated in the ATLAS top quark physics program, focused on the cross section measurement. 
I was one of the main analyzers for the first observation of top quark pair production and first cross section measurements with 2.9~pb$^{-1}$ of data at $\sqrt{s}=7$~TeV. 
I am a primary author of one of the analyses, involving the measurement of the cross section in the single lepton channel using a fit to the three-jet mass distribution. 
For this analysis I have authored many of the tools needed, such as a performant package to apply corrections to the MC and build the required distributions for detailed data-to-MC
comparisons. These tools then set the ground for all other Run-I analyses performed in the IFAE-Barcelona top group.
I have also developed techniques to estimate the multi-jet background in the analysis, to propagate systematic uncertainties from heavy flavor tagging calibrations, 
and calculate the signal significance. Beyond these important technical contributions, I was directly involved in the final cross section measurement.\\

\noindent [3] With 35~pb$^{-1}$ of data at $\sqrt{s}=7$~TeV I continued to be involved in the development of the $t\bar{t}$ cross section analysis optimizing the top mass reconstruction 
used in the template fit method to extract the cross section.\\

\noindent [4] Given the large number of collisions quickly delivered by LHC in 2011, I stopped my involvement in $t\bar{t}$ cross section measurement to focus on $4^{th}$ generation quark searches.
Since the beginning of the heavy top-like quark analysis, I have been the contact person in the collaboration. I am the main contributor of the first $t^{\prime}\bar{t^{\prime}} \rightarrow WbWb$ 
analysis in the single lepton channel in ATLAS with
1~fb$^{-1}$ of data at $\sqrt{s}=7$~TeV. I developed a user friendly interface to an already existing limit setting code that was used by about 10 published analysis in the 
ATLAS exotic group, and is still used by several analyses close to final Run-I publication. I was also editor of the internal documentation.\\

\noindent [5] With the full 4.9~fb$^{-1}$ dataset at $\sqrt{s}=7$~TeV, I was the main contributor to improving the result published in [4] using boosted $W$ bosons. 
I also developed all the methodology and tools to interpret the results in terms of vector-like quark, this analysis being the first one interpreting the results in a quasi model independent 
way. All my procedures and tools were adopted by all four other vector-like quarks analyses. 
I was also editor of the internal documentation.\\

\noindent [6] Previous vector-like top searches focused on $t^{\prime} \rightarrow Wb$, but after the observation of a Higgs boson with a mass of approximatively 125 GeV 
by the ATLAS and CMS collaborations, $t^{\prime} \rightarrow Hb$ became an important aspect of the vector-like top search strategy. 
For a SM-like Higgs boson, the dominant decay mode at this mass is through $b\bar{b}$, thus it is important to develop a new analysis sensitive to a large number of jets and $b-$jets,
in order to extend the sensitivity to vector-like quarks $t^{\prime} \rightarrow Ht, H \rightarrow b\bar{b}$. I was the main analyzer developing the $t^{\prime}\bar{t^{\prime}} \rightarrow Ht+X$ 
with 14.3~fb$^{-1}$ at $\sqrt{s}=8$~TeV. I was also editor of the internal documentation.\\

\noindent [7] With 14.3~fb$^{-1}$at $\sqrt{s}=8$~TeV, I was the main analyzer updating the result in [5] to search for  $t^{\prime}\bar{t^{\prime}} \rightarrow Wb+X$.
I performed the first ATLAS vector-like top combination of $t^{\prime}\bar{t^{\prime}} \rightarrow Ht+X$ and $ Wb+X$. I was also editor of the internal documentation.\\


\noindent [8] The $b^{\prime}\bar{b^{\prime}} \rightarrow Zb+X$ is an other fundamental aspect to complete the generic search for vector-like quarks.
I played a major role in this 8~TeV analyse by providing the tools and all my expertise in  understanding and interpreting the results in terms of vector-like quarks. \\

\noindent [9] With the full 2012 dataset of 20.3~fb$^{-1}$ I finalized the search for $T\bar{T} \rightarrow Wb+X$  for Run-I legacy publication being the main analyzer.
This preliminary result has been presented for the first time at the Moriond EW 2105 conference and represents the most stringent constraints to date on observed lower limits for pair produced heavy up type quarks.  \\

\noindent [10] After having performed various direct search for new physics I decided to focus on the precision measurement of the charge asymmetry in top quark pair production, 
measurement that indirectly probe hints of possible new physics contribution in top-pair production. I am leading this analysis, and results indicate that this measurement will be the 
most precise one at the LHC. I am now finalizing this analysis and editing the paper for internal review before publication in the coming two months.\\




\vskip 1.0 cm
{\bf \Large  References }\\
\vskip 1.0 cm

\begin{enumerate}
%[1]
\item
 {\bf ``Expected Performance of the ATLAS Experiment - Detector, Trigger and Physics''}
  \\{}G.~Aad {\it et al.}  [ATLAS Collaboration].
  \\{}arXiv:0901.0512 [hep-ex]

%[2]
\item 
  {\bf ``Measurement of the top quark-pair production cross section with ATLAS in pp collisions at $\sqrt{s}=7$ TeV''}
  \\{}G.~Aad {\it et al.}  [Atlas Collaboration].
  \\{}arXiv:1012.1792 [hep-ex]
  \\{}10.1140/epjc/s10052-011-1577-6
  \\{}Eur.\ Phys.\ J.\ C {\bf 71}, 1577 (2011)

%[3]
\item
  {\bf ``Measurement of the top quark-pair production cross section with ATLAS in the single lepton channel''}
  \\{}G.~Aad {\it et al.}  [ATLAS Collaboration].
  \\{}arXiv:1201.1889 [hep-ex]
  \\{}10.1016/j.physletb.2012.03.083
  \\{}Phys.\ Lett.\ B {\bf 711}, 244 (2012)

%[4]
\item
 {\bf ``Search for pair production of a heavy up-type quark decaying to a W boson and a b quark in the lepton+jets channel with the ATLAS detector''}
  \\{}G.~Aad {\it et al.}  [ATLAS Collaboration].
  \\{}arXiv:1202.3076 [hep-ex]
  \\{}10.1103/PhysRevLett.108.261802
  \\{}Phys.\ Rev.\ Lett.\  {\bf 108}, 261802 (2012)

%[5]
\item
  {\bf ``Search for pair production of heavy top-like quarks decaying to a high-$p_{\rm T}$ $W$ boson and a $b$ quark in the lepton plus jets final state at $\sqrt{s}=7$ TeV with the ATLAS detector''}
  \\{}G.~Aad {\it et al.}  [ATLAS Collaboration].
  \\{}arXiv:1210.5468 [hep-ex]
    \\{}10.1016/j.phcysletb.2012.11.071
\\{}Phys.\ Lett.\ B {\bf 718}, 1284 (2013) 

%[6]
\item
{\bf ``Search for heavy top-like quarks decaying to a Higgs boson and a top quark in the lepton plus jets final state in $pp$ collisions at $\sqrt{s}=$~8~TeV with the ATLAS detector''}
  \\{}The ATLAS Collaboration.
  \\{}ATLAS-CONF-2013-018
  
%[7]
\item
{\bf ``Search for pair production of heavy top-like quarks decaying to a high-$p_{\rm T}$ $W$ boson and a $b$ quark in the lepton plus jets final state in $pp$ collisions at $\sqrt{s}=$~8~TeV with the ATLAS detector''}
  \\{}The ATLAS Collaboration.
 \\{}ATLAS-CONF-2013-060

%[8]
\item
{\bf ``Search for pair and single production of new heavy quarks that decay to a $Z$ boson and a third generation quark in $pp$ collisions at $\sqrt{s}=~$8~TeV with the ATLAS detector''}
  \\{}G. Aad et al. (ATLAS Collaboration) (2014), 1409.5500
   \\{}arXiv:1409.5500 [hep-ex]
   \\{}JHEP {\bf 1411} (2014) 104
   
%[9]
%\item
%{\bf ``Search for anomalous production of events with same-sign dileptons and b jets in 14.3~fb$^{-1}$ of $pp$ collisions at $\sqrt{s}=$~8~TeV with the ATLAS detector''}
%  \\{}The ATLAS Collaboration.
%  \\{}ATLAS-CONF-2013-051

%[9]
 \item
{\bf "Search for production of vector-like quark pairs and of four top quarks in the lepton plus jets final state in $pp$ collisions at $\sqrt{s}=~$8~TeV with the ATLAS detector"}
\\{}The ATLAS Collaboration.
\\{}Presented at Moriond EW 2015, and to be published in JHEP

%[10]
\item
{\bf ``Measurement of the charge asymmetry in top quark pair production in pp collision data at $\sqrt{s}=$~8~TeV using the ATLAS detector''}
  \\{}The ATLAS Collaboration.
  \\{}In preparation (ATL-COM-PHYS-2014-087) and to be published in EPJC

%\cite{Golling:2016gvc}
\item
{\bf "Physics at a 100 TeV pp collider: beyond the Standard Model phenomena"}
  \\{}T.~Golling {\it et al.}
  \\{}CERN Yellow Report (2017) no.3,  441
  \\{}doi:10.23731/CYRM-2017-003.441
  \\{}[arXiv:1606.00947 [hep-ph]].
 
\end{enumerate}


\end{document}